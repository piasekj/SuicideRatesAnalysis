\documentclass[polish]{article}



\usepackage{polski}
\usepackage[T1]{fontenc}
\usepackage{tgpagella}
\usepackage{colortbl}
\usepackage[table]{xcolor}
% \usepackage[a4paper, total={7in, 10in}]{geometry}
\usepackage[a4paper, left=0.75in, right=0.75in, top=0.75in, bottom=0.75in]{geometry}
\usepackage{listings}
\usepackage{titlesec}
\usepackage{graphicx}
\usepackage[autostyle]{csquotes}
\DeclareQuoteAlias{dutch}{polish}
\usepackage[T1]{fontenc}
\usepackage[utf8]{inputenc}
\usepackage{babel}
\usepackage{soul}
\usepackage{indentfirst}
\usepackage{fancyhdr}
\usepackage{xcolor}
\usepackage{lipsum}
\usepackage{enumitem}
\usepackage{float}
\usepackage{amsmath}
\usepackage{hyperref}
\usepackage[normalem]{ulem}

\hypersetup{
    colorlinks=true,
    citecolor=black,
    filecolor=black,
    linkcolor=black,
    urlcolor=black
}

\setlength\parindent{24pt}


\titlelabel{\thetitle.\quad}


\lstset
{
    %basicstyle=\footnotesize,
    backgroundcolor=\color{black!5}, % set backgroundcolor
    basicstyle=\footnotesize\small,% basic font setting
    %basicstyle=\sffamily,
    numbers=left,
    stepnumber=0,
    showstringspaces=false,
    tabsize=1,
    breaklines=true,
    breakatwhitespace=false,
}

\lstnewenvironment{CPP}
  {\lstset{
    language=C++,
    backgroundcolor=\color{black!5}, % set backgroundcolor
    basicstyle=\footnotesize\small,% basic font setting
    commentstyle=\color{green!60!black},
    keywordstyle=\color{magenta},
    stringstyle=\color{blue!50!red},
    numberstyle=\footnotesize\color{gray},
    numbersep=10pt,
    %stepnumber=2,
    frame=L,
    %framerule=1pt,
    %rulecolor=\color{red},
    breaklines=true,
    postbreak=\mbox{\textcolor{red}{$\hookrightarrow$}\space},
    numbers=none,
    showstringspaces=false,
    tabsize=1,
    breakatwhitespace=false,
    inputpath=code}}
  {}

  \lstnewenvironment{C}
  {\lstset{
    language=C,
    backgroundcolor=\color{black!5}, % set backgroundcolor
    basicstyle=\footnotesize\small,% basic font setting
    commentstyle=\color{green!60!black},
    keywordstyle=\color{magenta},
    stringstyle=\color{blue!50!red},
    numberstyle=\footnotesize\color{gray},
    numbersep=10pt,
    %stepnumber=2,
    frame=L,
    %framerule=1pt,
    %rulecolor=\color{red},
    breaklines=true,
    postbreak=\mbox{\textcolor{red}{$\hookrightarrow$}\space},
    numbers=none,
    showstringspaces=false,
    tabsize=1,
    breakatwhitespace=false,
    inputpath=code}}
  {}

  \lstnewenvironment{Python}
  {\lstset{
    language=Python,
    backgroundcolor=\color{black!5}, % set backgroundcolor
    basicstyle=\footnotesize\small,% basic font setting
    commentstyle=\color{green!60!black},
    keywordstyle=\color{magenta},
    stringstyle=\color{blue!50!red},
    numberstyle=\footnotesize\color{gray},
    numbersep=10pt,
    %stepnumber=2,
    frame=L,
    %framerule=1pt,
    %rulecolor=\color{red},
    breaklines=true,
    postbreak=\mbox{\textcolor{red}{$\hookrightarrow$}\space},
    numbers=none,
    showstringspaces=false,
    tabsize=1,
    breakatwhitespace=false,
    inputpath=code}}
  {}


\newenvironment{cverbatim}
 {\SaveVerbatim{cverb}}
 {\endSaveVerbatim
  \flushleft\fboxrule=0pt\fboxsep=.5em
  \colorbox{cverbbg}{\BUseVerbatim{cverb}}%
  \endflushleft
}

\newenvironment{lcverbatim}
 {\SaveVerbatim{cverb}}
 {\endSaveVerbatim
  \flushleft\fboxrule=0pt\fboxsep=.5em
  \colorbox{cverbbg}{%
    \makebox[\dimexpr\linewidth-2\fboxsep][l]{\BUseVerbatim{cverb}}%
  }
  \endflushleft
}

\newcommand{\ctexttt}[1]{\colorbox{cverbbg}{\texttt{#1}}}
% \newverbcommand{\cverb}
%   {\setbox\verbbox\hbox\bgroup}
%   {\egroup\colorbox{cverbbg}{\box\verbbox}}


%%%%%%%%%%%%%%%%%%%%%%%%%%%%%%%%%%%%%%%%%%%%%%%%%%%%%%%%%%%%%%%%%%%%%%%%%%%%%%%
% WEII STOPKA
%%%%%%%%%%%%%%%%%%%%%%%%%%%%%%%%%%%%%%%%%%%%%%%%%%%%%%%%%%%%%%%%%%%%%%%%%%%%%%%
% \pagestyle{fancy}
% % \fancyhf{}

% % Upper header line configuration
% \renewcommand{\headrulewidth}{0pt}
% \chead{}
% \lhead{}
% \rhead{}


% % Header configuration
% \fancyfoot[C]{%
%     \begin{minipage}{\textwidth}
%         \centering
%         \begin{minipage}{0.1\textwidth}
%             \includegraphics[width=\textwidth]{img/zsz_logo.png}
%         \end{minipage}%
%         \hspace{0.02\textwidth}%
%         \begin{minipage}{0.85\textwidth}
%             \raggedright
%             \textcolor[rgb]{0.5,0.5,0.5}{
%                 \textbf{ZAKŁAD SYSTEMÓW ZŁOŻONYCH}\\
%                 Wydział Elektrotechniki i Informatyki\\
%                 ul. Wincentego Pola 2, 35-959 Rzeszów, tel. 17 865 1340\\
%                 \texttt{zsz.prz.edu.pl}
%             }
%         \end{minipage}
%     \end{minipage}
% }




\begin{document}

    \LARGE\begin{titlepage}

        \begin{center}

        \includegraphics*{img/wmifs_pl.png}



        \vspace{3cm}

        \textbf{Statystyczna Analiza Danych}

        \vspace{1cm}
            Analiza samobójstw *NAZWA DO ZMIANY* % TODO

        \vspace{5cm}

        \raggedleft\vfil
        Jakub Piasek\\
        169828 \\
        Inżynieria i Analiza Danych II rok, gr. lab. nr 1\\
        30.05.2025


        \end{center}

    \end{titlepage}

    \normalsize

    \tableofcontents

    \newpage

    \section{Wstęp do projektu}

    \subsection{Opis użytych danych}

    Dane wykorzystane w analizie pochodzą ze zbioru \textbf{„Suicide Rates Overview 1985 to 2016”}, dostępnego na platformie \textbf{\href{https://www.kaggle.com/datasets/russellyates88/suicide-rates-overview-1985-to-2016}{\textcolor{blue}{\uline{Kaggle}}}}. Zbiór ten został opracowany na podstawie informacji z czterech źródeł międzynarodowych:

    \begin{itemize}
        \item \textbf{{\href{http://www.who.int/mental_health/suicide-prevention/en/}{\textcolor{blue}{\uline{World Health Organization (WHO)}}}}} – dane dotyczące samobójstw na całym świecie,
        \item \textbf{{\href{http://databank.worldbank.org/data/source/world-development-indicators#}{\textcolor{blue}{\uline{World Bank}}}}} – wskaźniki gospodarcze, w tym PKB (GDP) per capita,
        \item \textbf{{\href{http://hdr.undp.org/en/indicators/137506}{\textcolor{blue}{\uline{United Nations Development Program (UNDP)}}}}} – Wskaźnik Rozwoju Społecznego (HDI),
        \item \textbf{\href{https://www.kaggle.com/szamil/suicide-in-the-twenty-first-century/notebook}{\textcolor{blue}{\uline{Szamil’s Kaggle dataset}}}} – zawierający podstawowe dane o samobójstwach według kraju, płci, grupy wiekowej i roku.
    \end{itemize}

    Dane obejmują okres od \textbf{1985 do 2016 roku} i zawierają informacje dla ponad \textbf{100 krajów}. Główne zmienne to m.in.:

    \begin{itemize}
        \item liczba samobójstw,
        \item liczba ludności,
        \item współczynnik samobójstw (na 100 tys. mieszkańców),
        \item rok, kraj, płeć, grupa wiekowa,
        \item PKB per capita (USD),
        \item HDI.
    \end{itemize}

    Wskaźnik samobójstw w latach 1985–2016 porównywany jest ze statystykami społeczno-gospodarczymi, co pozwala analizować zależności pomiędzy rozwojem gospodarczym a częstością samobójstw w podziale na kraje i lata (Compares socio-economic info with suicide rates by year and country).

    \subsection*{Uzasadnienie wyboru danych}

    Zbiór danych został wybrany ze względu na:

    \begin{itemize}
        \item \textbf{dużą szczegółowość czasową i demograficzną}, co umożliwia analizę trendów samobójstw w różnych grupach społecznych;
        \item \textbf{powiązanie danych zdrowotnych z wskaźnikami ekonomicznymi i rozwojowymi}, co pozwala badać korelacje między poziomem rozwoju kraju a częstością samobójstw;
        \item \textbf{globalny zasięg} – dane obejmują kraje o różnym poziomie rozwoju, co umożliwia porównania międzynarodowe;
        \item \textbf{wiarygodność źródeł} – dane pochodzą z renomowanych organizacji międzynarodowych (WHO, World Bank, UNDP).
    \end{itemize}

    Zbiór ten jest szczególnie przydatny w badaniach nad zdrowiem psychicznym, polityką społeczną i profilaktyką samobójstw, zarówno na poziomie globalnym, jak i krajowym.


    \newpage

    \section{Podstawowe parametry opisowe}

    \newpage

    \section{Wykresy}

    \newpage

    \section{Hipotezy}

    \newpage

    \section{Wnioski}

\end{document}
